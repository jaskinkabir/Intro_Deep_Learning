\documentclass{article}
%\usepackage[margin=1in]{geometry}
\usepackage{graphicx} % Required for inserting images
\usepackage{hyperref}
\usepackage{amsmath}
\usepackage{titling}
\usepackage{enumitem}
\usepackage{makecell}
\usepackage{minted}
\usepackage{url}
\usepackage{tabularx}
\usepackage{graphicx}
\renewcommand\maketitlehooka{\null\mbox{}\vfill}
\renewcommand\maketitlehookd{\vfill\null}

\begin{document}
\centering

\title{\Huge Intro Deep Learning Homework 6}

\author{ \huge
Jaskin Kabir \\
\Large Student Id: 801186717 \\
\Large \href{https://github.com/jaskinkabir/Intro_Deep_Learning/tree/master/HM6}{GitHub:}\\\url{https://github.com/jaskinkabir/Intro_Deep_Learning/tree/master/HM6}
}

\date{April 2025}

\begin{titlingpage}
\maketitle
\end{titlingpage}
\raggedright

\section{Problem 1: ViT From Scratch}
\subsection{Model Architectures}
This section focuses on training four
variations of a vision transformer based
network for image classification on the
CIFAR-100 Dataset. The four models shared the
same parameters other than the patch size and
number of attention layers.

The shared parameters are as follows:
\begin{itemize}
    \item Embedding Size: 192
    \item MLP Hidden Size: 384
    \item Number of Attention Heads: 4
    \item Classifier Head Hidden Layers: 384, 192
\end{itemize}
The four variations are combinations of a
patch size of 4 or 8 and either 4 or 8
attention layers. The models were trained on
the CIFAR-100 dataset with a batch size of
64, and learning rate of 5e-4. They were also
compared to a baseline ResNet-18 pretrained
on ImageNet.  The Resnet-18 model was
fine-tuned for CIFAR-100 for 10 epochs with a
batch size of 64 and a learning rate of 1e-3.

\subsection{Results}
\begin{table}[h]
    \centering % Center the table
    \begin{tabular}{|c|c|c|c|c|}
        \hline
        \textbf{Model} & \textbf{Parameters} & \textbf{MACs} & \textbf{Avg Epoch Time} & \textbf{Accuracy}  \\
        \hline
        \textbf{ViT Patch4 Attn4} & 1,377,508 & 756,480 & 11.76s & 43.9\% \\
        \hline
        \textbf{ViT Patch4 Attn8} & 2,565,604 & 756,480 & 15.07s & 44.6\% \\
        \hline
        \textbf{ViT Patch8 Attn4} & 1,395,940 & 756,480 & 10.58s & 37.9\% \\
        \hline
        \textbf{ViT Patch8 Attn8} & 2,584,036 & 756,480 & 11.53s & 37.7\% \\
        \hline
        \textbf{ResNet-18} & 11,227,812 & 1,816,096,768 & 30.07s & 56.9\% \\
    \end{tabular}
    \caption{ViT Model Comparisons}
    \label{tab:complexities}
\end{table}

The vision transformer models were soundly
outperformed by the ResNet-18 model. This is
due to a number of reasons. Firsly, the
Resnet is much more complex in terms of
parameters and MACs. The ResNet-18 model has
11 million parameters and 1.8 billion MACs,
while the largest ViT model has only 2.5
million parameters and 756 million MACs. This
means that the ResNet-18 model is able to
learn more complex features from the data,
which is important for image classification
tasks. Additionally, the vision transformers
were trained for 50 epochs on the CIFAR-100
dataset, while the ResNet-18 model was
trained on the much larger ImageNet dataset.
With their lack of inductive bias, the vision
transformers were unable to learn the
features of the dataset as well as the
ResNet-18 model.

\section{Problem 2: Pretrained Swin Transformers}
Microsoft's Swin Transformer is a vision
transformer model trained on the ImageNet
dataset. For this experiment, the large and
tiny models were loaded and fine-tuned for
the CIFAR-100 dataset for 5 epochs. The training used a batch size of 32 and a 






\end{document}